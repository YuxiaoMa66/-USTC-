\documentclass[a4paper,AutoFakeBold,AutoFakeSlant]{ctexart}
\usepackage[a4paper,left=3cm,right=3cm,top=2.5cm,bottom=2.5cm]{geometry}
\usepackage{graphicx}
\usepackage{pythonhighlight}
\usepackage[mathscr]{eucal}
\usepackage{mathrsfs}
\usepackage{booktabs}
\usepackage{capt-of} 
\usepackage{hyperref} 
\usepackage{abstract}
\usepackage{amsmath}
\usepackage{listings}
\usepackage{color}
\usepackage{caption}
\usepackage{subfigure}
\usepackage{float}

% -- 中文字体 --
%\setCJKmainfont{Microsoft YaHei}  % 微软雅黑
%\setCJKmainfont{YouYuan}  % 幼圆
%\setCJKmainfont{NSimSun}  % 新宋体
%\setCJKmainfont{KaiTi}    % 楷体
% \setCJKmainfont{SimSun}   % 宋体
%\setCJKmainfont{SimHei}   % 黑体
 
% -- 英文字体 --
\setmainfont{Times New Roman}
% \setmainfont{DejaVu Sans}
% \setmainfont{Latin Modern Mono}
% \setmainfont{Consolas}

\usepackage{xcolor}  	%高亮使用的颜色
\definecolor{commentcolor}{RGB}{85,139,78}
\definecolor{stringcolor}{RGB}{206,145,108}
\definecolor{keywordcolor}{RGB}{34,34,250}
\definecolor{backcolor}{RGB}{220,220,220}

\usepackage{accsupp}	
\newcommand{\emptyaccsupp}[1]{\BeginAccSupp{ActualText={}}#1\EndAccSupp{}}

\usepackage{listings}
\lstset{						%高亮代码设置
	language=python, 					%Python语法高亮
	linewidth=0.95\linewidth,      		%列表list宽度
	%basicstyle=\ttfamily,				%tt无法显示空格
	commentstyle=\color{commentcolor},	%注释颜色
	keywordstyle=\color{keywordcolor},	%关键词颜色
	stringstyle=\color{stringcolor},	%字符串颜色
	%showspaces=true,					%显示空格
	numbers=left,						%行数显示在左侧
	numberstyle=\tiny\emptyaccsupp,		%行数数字格式
	numbersep=5pt,						%数字间隔
	frame=single,						%加框
	framerule=0.1pt,						%划线
	escapeinside=@@,					%逃逸标志
	emptylines=1,						%
	xleftmargin=3em,					%list左边距
	backgroundcolor=\color{backcolor},	%列表背景色
	tabsize=4,							%制表符长度为4个字符
	% gobble=4							%忽略每行代码前4个字符
}




\renewcommand{\abstractname}{}    % clear the title
\renewcommand{\absnamepos}{empty}
%去除摘要两边缩进
\makeatletter
  \renewenvironment{abstract}{%
      \if@twocolumn
        \section*{\abstractname}%
      \else
        \small
        \begin{center}%
          {\bfseries \abstractname\vspace{-.5em}\vspace{\z@}}%
        \end{center}%
      \fi}
      {}
  \makeatother
  \lstset{
    language=Matlab,
    keywords={break,case,catch,continue,else,elseif,end,for,function,
       global,if,otherwise,persistent,return,switch,try,while},
    basicstyle=\ttfamily,
    keywordstyle=\color{blue}\bfseries,
    commentstyle=\color{dkgreen},
    stringstyle=\color{dkpurple},
    backgroundcolor=\color{white},
    tabsize=4,
    showspaces=false,
    showstringspaces=false
 }




\title{\heiti{深度学习Lab1实验报告}}
\author{PB19151769~~~~~~马宇骁}

\begin{document}

\maketitle

\begin{abstract}\zihao{-4} \kaishu
\noindent 
\textbf{\heiti 摘要:}简单描述fine tuning(dropout、normalization、learning rate decay、residual 
connection、网络深度等), 最后的test准确率,对比,以及Loss随着epoc改变的曲线。
   \newline
\textbf{\heiti 关键词:}深度学习,CNN,ResNet
\end{abstract}


\section{实验目标}
使用pytorch或者tensorflow实现卷积神经网络,在ImageNet数据集上进行图片分类。研究dropout、
normalization、learning rate decay、residual connection、网络深度等超参数对分类性能的影响。



\section{网络模型代码结构}


\subsection{CNN}
\subsubsection{简述}
卷积神经网络主要由这几类层构成:输入层、卷积层,ReLU层、池化(Pooling)层和全连接层。
通过将这些层叠加起来,就可以构建一个完整的卷积神经网络。

由于参数共享,每个滤波器包含 $F*F*D$ 个权重,卷积层一共有 $F*F*D_1*K$ 个权重和 K 个偏置。
在输出数据体中,第d个深度切片(空间尺寸是 $W_2*H_2$ ),用第d个滤波器和输入数据进行有效卷
积运算的结果(使用步长S),最后在加上第d个偏差。

在连续的卷积层之间会周期性地插入一个池化层。它的作用是逐渐降低数据体的空间尺寸,
这样的话就能减少网络中参数的数量,使得计算资源耗费变少,也能有效控制过拟合。

穿插使用归一化,ReLU,Dropout(在训练过程的前向传播中,让每个神经元以一定概率p处于不激活的状态。以达到减少过拟合的效果。),
最后全连接完成整个CNN。
\begin{itemize}
  \item $ INPUT \rightarrow [CONV \rightarrow RELU \rightarrow CONV \rightarrow RELU \rightarrow POOL]*n_1 \rightarrow [FC \rightarrow RELU]*n_2 \rightarrow FC   $
\end{itemize}

\subsubsection{实现}
根据书上LeNet-5和AlexNet的方法考虑借鉴其中的算法,使用5*5的卷积核,2*2的池化层,发现一次train的准确率太低不足0.5个百分点;然后考虑使用5*5的卷积层,步长2,
零填充等类似后者的5层卷积层2层全连接层(AlexNet为3层,但执行速度太慢),结果改进并不显著,只达到0.7-0.9的准确率百分点。于是进行多次尝试,例如:

\textsc{Python}程序显示如下:
\begin{python}
  ####################### task 1.1 ##########################
  self.relu0 = nn.ReLU()
  self.mp = nn.MaxPool2d(2) # 池化之后的size变成(32*32*32)
    # O = (I - K + 2P)/ S +1

  self.conv1 = nn.Sequential(nn.Conv2d(32, 64, 5,stride=1,padding=0),
                              nn.Dropout(0.5),
                              nn.BatchNorm2d(64),                         
                              nn.ReLU(),
                              nn.MaxPool2d(2)) # \rightarrow (14*14*64)
  self.conv2 = nn.Sequential(nn.Conv2d(64, 512, 3,stride=1,padding=1), 
                              nn.Dropout(0.5),
                              nn.BatchNorm2d(512),
                              nn.ReLU(),
                              nn.MaxPool2d(2)) #  -> (7*7*512)
  self.conv3 = nn.Sequential(nn.Conv2d(512,512,2,stride=1,padding=0), 
                              nn.Dropout(0.5), 
                              nn.BatchNorm2d(512),
                              nn.ReLU(),
                              nn.MaxPool2d(2)) #  (3*3*512)
  self.dense = nn.Sequential(nn.Linear(3*3*512, 3*3*512),
                              nn.ReLU(),
                              nn.Dropout(0.5),
                              nn.Linear(3*3*512,200))  
#发现这样准确率还是太低了emmm
\end{python}
最终在查阅资料-尝试的反复循环中,使用如下CNN模型加上助教的第一层,共6层网络,其单次train的准确率可以到2.3-8.9个百分点。

\textsc{Python}程序显示如下:
\begin{python}
  self.conv1 = nn.Conv2d(32, 64, 2, stride = 1, padding=1)
  self.bn1 = nn.BatchNorm2d(64)

  self.conv2 = nn.Conv2d(64, 192, 2, stride=1, padding=1)
  self.bn2 = nn.BatchNorm2d(192)

  self.conv3 = nn.Conv2d(192, 256, 2, stride=1, padding=1)
  self.bn3 = nn.BatchNorm2d(256)
  
  self.conv4 = nn.Conv2d(256, 256, 3, stride=1, padding=1)
  self.bn4 = nn.BatchNorm2d(256)

  self.maxpool = nn.MaxPool2d(3, 2)

  # self.fc1 = nn.Linear(192*3*3, 192*3*3)
  self.fc2 = nn.Linear(256*3*3, 200)
\end{python}


\subsection{ResNet}
\subsubsection{简述}
ResNet是解决了深度CNN模型难训练的问题。
对于一个堆积层结构(几层堆积而成)当输入为 $x$ 时其学习到的特征记为 $H(x)$ ,
现在我们希望其可以学习到残差 $F(x) = H(x) - x$ ,这样其实原始的学习特征是 $F(x)+x$ 。
之所以这样是因为残差学习相比原始特征直接学习更容易。当残差为0时,此时堆积层仅仅做了恒等映射,
至少网络性能不会下降,实际上残差不会为0,这也会使得堆积层在输入特征基础上学习到新的特征,从而拥有更好的性能。

\subsubsection{实现}
由于残差网络学习的更好的特性,因此,在构建补全时,加上助教的一层使用6层网络,对于优化单次train的准确率就相比CNN容易,采用如下5+1层残差网络的
第一次尝试就突破2个百分点大关,就不再继续修改折磨。

\textsc{Python}程序显示如下:
\begin{python}
  ######################## task 2.2 ##########################
  self.conv1 = nn.Sequential(nn.MaxPool2d(kernel_size=3,stride=2,padding=1),
                             ResNetLayer(64,64,False),
                             ResNetLayer(64,64,False),
                             ResNetLayer(64,64,False))
  self.conv2 = nn.Sequential(ResNetLayer(64,128,True),
                             ResNetLayer(128,128,False),
                             ResNetLayer(128,128,False),
                             ResNetLayer(128,128,False))
  self.conv3 = nn.Sequential(ResNetLayer(128,256,True),
                             ResNetLayer(256,256,False),
                             ResNetLayer(256,256,False),
                             ResNetLayer(256,256,False))
  self.conv4 = nn.Sequential(ResNetLayer(256,512,True),
                             ResNetLayer(512,512,False),
                             ResNetLayer(512,512,False))
  self.avg_pool = nn.AdaptiveAvgPool2d(1)
  self.fc = nn.Linear(512, 200)
\end{python}



\section{训练模型完成及分析}

\subsection{综述}

训练模型要求通过读取data数据集里的train的90000张训练图片进行图片学习,最后预测test的10000张图片来测试预测的模型的准确率。

考虑到在CNN模型单次运行train函数在本人所使用的笔记本电脑(HUAWEI matebookX Pro2019,显卡MX250)所需时间15分钟,第一次尝试预测运行了7个小时且因为模型
优化不佳准确率偏低整个事件过于离谱,
因此,采取用59元租用两块gtx1060的服务器一天来训练模型。

\subsubsection{补全及改进}

对于助教所提供的train的代码框架,发现只判断是否将模型放在gpu上训练,于是,考虑修改为使用两块gpu进行并行计算加速训练
(实际发现并行计算确实能加速大约180-190$\%$),如下:

\textsc{Python}程序显示如下:
\begin{python}
  # model = CNN()
  model = ResNet()
  model.apply(init_weights)

  # if gpu:
  #     model.cuda()

  device = torch.device("cuda:0" if torch.cuda.is_available() else "cpu")
  if torch.cuda.device_count() > 1:
      model = nn.DataParallel(model)
  model.to(device) #用两块1060加速
\end{python}

然后预想设置maxepoch为40,学习速率0.015.此时,构建一个优化器对象Optimizer,用来保存当前的状态,并能够根据计算得到的梯度来更新参数;
之后再利用ReduceLROnPlateau来更新学习率,用min模式检测metric是否不再减小,patience用10作为不再减小的累计次数,最小的允许lr
设置为1e-5,其他的参数均为默认值;最后设置使用交叉熵损失函数作为每次的交叉损失的计算方式。

对于validate,test函数,检测准确性,不用.train(),故采取与train类似的方式完成。

对于fit函数,需要判断是否需要提前退出,即early$\_$stop$\_$counter大于patience的时候,不再迭代,更新模型权重之后退出循环,返回相应值。期间保留准确性和损失,方差历史。

最后的模型测试之前需要先将模型储存,再继续将单gpu改为并行计算进行测试,保留历史。

最后将历史进行作图,得到曲线。

\subsection{CNN}

使用自己编写的CNN模型,将fit函数的调用参数中的maxepoch设置为40,单开GPU时发现一个单次trian需要大约1.5分钟,双开并行计算时大约48秒。
最终,在40次迭代之后,发现最后输出如下:
\begin{itemize}
  \item 29.099999999999998
\end{itemize}

将模型保存,进行测试,结果如下:
\begin{itemize}
  \item Test Loss:  0.0496  $|$  Test Accuracy: 28.6700
\end{itemize}

大于20个百分点,可以认为效果不错。将Loss 和Accuracy随着epoc 改变的曲线以及方差曲线展示如下:

\begin{figure}[H]
  \centering
  \includegraphics[scale=0.6]{cnn.png}
  \caption{CNN}
  \label{}
\end{figure}

\subsection{ResNet}

由于发现在一次train单开GPU需要6分钟,并行计算也需要3分半,因此在ResNet网络的fit参数中的maxepoch设置为15减小运行总时间。
最终,训练结果最后输出如下:
\begin{itemize}
  \item 23.400000000000002
\end{itemize}

可以肯定,残差网络的训练预测模型确实比普通的卷积网络优秀,在训练次数明显小的情况下也能得到较为良好的模型训练结果。
模型保存测试结果如下:
\begin{itemize}
  \item Test Loss:  0.0587  $|$  Test Accuracy: 22.8100
\end{itemize}

将Loss 和Accuracy随着epoc 改变的曲线以及方差曲线展示如下:

\begin{figure}[H]
  \centering
  \includegraphics[scale=0.6]{res.png}
  \caption{ResNet}
  \label{}
\end{figure}


\end{document}
